\section{Introduction} \label{intro}
\subsection{BSS Preview}
Imagine that two people are speaking simultaneously in a room. There are two microphones which are in different locations and record the stereo signals generated by two people. Assuming that the each of the recorded time signals $x_1(t)$ and $x_2(t)$ is a linear combination of the speeches $s_1(t)$ and $s_2(t)$. We could express this as a set of linear equations:

\begin{align}
    x_1(t) = a_{11}s_1(t) + a_{12}s_2(t)\\
    x_2(t) = a_{21}s_1(t) + a_{22}s_2(t)
\end{align}

where parameter $a_{ij}$ depends on the distance of the microphones from the speakers. Time delays or other extra factors are ignored temporarily from our simplified mixing model. It would be useful in real-world applications if we can estimate the original speech signals $s_1(t)$ and $s_2(t)$ using only the recorded signals $x_1(t)$ and $x_2(t)$. This is addressed as the well-known cocktail party problem and also the chief motivation behind blind source separation. The similar situation is also common in telecomms, medical signal and image processing. If we knew the parameters $a_{ij}$ , we find $s_1(t)$ and $s_2(t)$  by solving linear equations or matrix inversion. The problem is, however we have no information about $a_{ij}$. the problem is considerable difficult.\\

The lack of prior knowledge of the mixing process can be compensated by a statistically strong but often physical plausible assumption of independence between the source signals. Independent Component Analysis (ICA) was first proposed to solve the cocktail party problem. The goal of ICA is to determine the original sources given mixtures of those sources, assuming that the sources are statistically independent and non Gaussian. Derivatives of the classical ICA methods includes JADE \cite{JADE720250}, FastICA \cite{fastICA777510}. Generally speaking, ICA algorithms are about devising adequate contrast functions which are related to approximation of independence\cite{HYVARINEN2000411}. However ICA is limited to the determined BSS problem when we have equal number of mixtures and the number of sources. This is because we need to find the inverse of the mixing matrix while optimising the contrast function in ICA. But only square matrix has such a inverse. 


\subsection{BSS by Sparsity}
Although ICA is proved to be effective in many BSS applications. The statistical independence assumption in the time domain cannot be applied to all scenarios. Sparsity-based approaches have drawn much attention in recent years. The term sparse refers to signals with small number of nonzeros with respect to some representation bases \cite{ZibulevskyMichael2001BSSb}. More specifically, sources have mutually disjoint support sets in a dictionary. This is exploited for instance in Sparse Component Analysis (SCA) \cite{SCA2005}. In SCA we make assumption that the sources to be unmixed can be sparsely represented in a predefined common basis or dictionary (for instance, a wavelet frame). A two-step approach \cite{BOFILL20012353} was proposed  to solve the underdetermined BSS problem using sparsity, in which the mixing system is first estimated using clustering methods, then the sources are estimated thanks to pursuit methods (e.g. basis pursuit, matching pursuit).\\

In many cases, basis pursuit or matching pursuit synthesis algorithms are computationally quite expensive. Furthermore, the traditional SCA requires highly sparse signals. Unfortunately, this is not the case for high dimensional signals and especially in image processing. We present in this report an alternative to these approaches, the Morphological Component Analysis (MCA) \cite{BobinJ_2007SaMD, BobinJ_2006Mdas} is a method which sources can be sparsely represented using several different dictionaries. For example, images normally contains contour and texture, the former is well sparsified using curvelets tight frame whereas the latter may be well represented using local cosine transform (DCT). 
%Extensions of MCA have been proposed to solve the case when each distinct source can be sparse representated in a specific dictionary. 
Multichannel morphological component analysis (MMCA) \cite{Starck2005MorphologicalCA} and generalised morphological component analysis (GMCA) are extentions of MCA to the multichannel case. In MMCA setting, we assume that the sources have strictly different morphologies (i.e., each source is assumed to be sparsely representated in one particular orthonormal basis). In GMCA, each source is modeled as the linear combination of a number of morphologial components where each component is sparse in a specific orthonormal basis. As mentioned in the Abstract, the morphological componene analysis currently does not solve the underdetermined BSS problem. Hence, in this project, our work will be carried out under the determined BSS framework.

\subsection{Structure of Report}
In Chapter \ref{background} of this report. We first provide sufficient background knowledge for blind source separation problem setting. 
Real world applications of BSS will be introduced before the performance measurements to evaluate different BSS algorithms are defined. 
Moreover, we will discuss the the well-established ICA algorithm as it is selected as our baseline method.
We then turn our discussion to sparsity and morphological diversity. The idea of overcomplete dictionaries is followed. 
We then look in to multichannel morphological component Analysis and generalised morphological component analysis.
%In addition adaptive dictionary learning is introduced.
Finally, we look into the future research trends in blind source separation.
Chapter \ref{inplementation} shows a self proposed research plan which tells a possible research direction for my future work. A contingency plan and risk assessment are also presented in this chapter.
The evidence of ethical, legal and safety issues is shown in Chapter \ref{ethical}. Conclusions and perspectives are presented at the end.\\

This chapter has considered the BSS methods which will be used in next stage of this project. Firstly, the linear instantaneous model was introduced and BSS assumptions were stated. We then introduced performance metrics to evaluate the performances of various BSS algorithms. The problem was then extended to the case of sparsity where morphological component analysis is discussed. Finally, we foresee the future research trends in morphological component analysis based blind source separation. 
