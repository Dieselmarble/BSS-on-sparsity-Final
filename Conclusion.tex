\section{Conclusion and futurework}
In this report, different approaches for blind source separation are investigates and discussed. Then a sparse clustering based block dictionary learning algorithm is applied to the BSS problem. In every iteration of the BSS process, the proposed algorithm repeat two stages, a block structure clustering step (SAC) and a dictionary update setp (BK-SVD). When the maximal block size in SAC is reduced to 1, the proposed algorithm reverts to normal K-SVD. In contrasting to the normal K-SVD dictionary learning BSS algorithm, the proposed one is noted to give a sparser representation of the target image and exhibit a better estimation of the mixing matrix and sources.\\

However the proposed methods has certain limitations. The computation cost of our proposed method is not satisfactory. This is due to the cubed complexity of SAC algorithm and blockwise updating manner of BK-SVD algorithm. In the future, we may consider the SimCo method \cite{6340354} in computing the dictionary atoms. SimCo allows updating all codewords and all sparse coefficients simultaneously and is expected to significantly speed up the learning process. To further improve the proposed adaptive BSS methods, one could try and make the dictionary learning step less susceptible to local minimum traps. In addition, training one dictionary to sparsely represent all the sources is an alternative to calculating multiple distinct dictionaries, as long as the dictionary redundancy is large enough. An obvious advantage of using one dictionary is that the computational cost does not increase when the number of sources increases. Another refinement could be replacing blocks in the dictionary that contributes little to signal representaions with the least significant signal elements. This is expected to further improve the reconstruction ability of out dictionary learning algorithm.  

