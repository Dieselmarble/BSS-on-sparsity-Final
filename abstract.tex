\begin{abstract}
\noindent
Blind source separation (BSS) is emerging techniques in array processing and data analysis that aim to recover unobserved sources from observed mixtures, where the mixing matrix is unknown. The classical ICA methods have limitation that the sources have to be mutually independent and the mixing matrix have to be full-column rank. To overcome these limitaions, we introduce sparsity based methods which decompose the source signal sparsely in a dictionary. Morphological component analysis (MCA) theory is proposed based on theory of sparse representation. It assumes that the signal is a linear combination of several components having different geometry, and each embodiment of the component can be sparsely represented in a dictionary, and not sparsely represented in others. In recent years, this theory has been applied to solve the blind source separation problem and obtained good results. As MCA are currently applied to determined mixtures solely. Therefore, the multichnnel case with more observations than sources is considered in this report.\\


The objectives of this report are to review some of the key approaches that have been developed to address the BSS problem, and to further discuss sparsity based methods in blind source separation. It first describes the theory behind sparse representation and sparse decomposition algorithms. This report then gives a decomposition algorithm based on block coordinate relaxation morphological component analysis whose variants have been applied to generalised morphological component analysis (GMCA) and multichannel morphological component analysis (MMCA). In the future we aim to introduce several common sparse representation dictionaries for the image processing of the cartoon and texture parts. By the end of the final report, we are expected to perform simulations on GMCA, MMCA and the baseline FastICA method. In addition, we are expected to propese our improved morphological component analysis algorithm by imposing further constraints on the components in sources. 
\end{abstract}
\newpage