\begin{abstract}
\noindent
Blind source separation (BSS) is a prevalent technique in array processing and data analysis that aims to recover unknown sources from observed mixtures, where the mixing matrix is unknown. The classical ICA methods requires prior statistical knowledge that the sources have to be mutually independent. In order to overcome the limitations, sparsity based methods are introduced, which decompose the source signal sparsely in a prescribed dictionary. Morphological component analysis (MCA) theory is proposed based on theory of sparse representation. It assumes that the signal is a linear combination of several components having different geometry, and each embodiment of the component can be sparsely represented in a dictionary, and not sparsely represented in others. In recent years, this theory has been applied to solve the blind source separation problem and obtained good results.\\

The objectives of this report are to review some of the key approaches derived from the classical ICA method (JADE, EFICA) that have been developed to address the BSS problem, and to further discuss sparsity based methods in blind source separation. It first describes the theory behind sparse representation and sparse decomposition algorithms. Then we aim to introduce several common sparse representation dictionaries for the image processing of the cartoon and texture parts, after which this report gives a decomposition algorithm based on block coordinate relaxation morphological component analysis whose variants have been applied to the multichannel morphological component analysis (MMCA) and generalised morphological component analysis (GMCA).\\

In the implementation part, we are expected to perform image segmentation experiment and blind image source separation experiment using the techniques we have introduced. We then apply the BSS techniques to digital warkermarking which employs the idea of sparsity to build an image encryption and decryption system and to benchmark against the prevailing least significant bit (LSB) warkermarking methods.
\end{abstract}
\newpage