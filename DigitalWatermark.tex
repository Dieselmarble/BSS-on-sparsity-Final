\section{Morphological decomposition in Steganography}
With the explosive growth of information on the internet, how to effectively protect the author's copyright cause increasing concerns. As an efficacious authentication method, digital watermark has been widely applied to the field of information security. Digital watermarking methods can be divided into two major streams: spatial method and transformed domain methods. The former is easy to implement but has drawbacks on its robustness to attack. The latter involves modulating the watermarking signal onto transformed domains (e.g. Fourier, Wavelet, DCT), it is often related to steganography to achieve better invisibility. Steganography is the practice of concealing a message within the host media. Stegnography of digital watermark can improve the communication security because the existence of embedding digital watermark is unknown to attackers, regardless it is intentional or unintentional. In the following sections, we will discuss the DCT transformed domain digital watermarking techniques. In addition, for originality, we combine the idea of \textit{Morphological Component Analysis} with digital watermark and propose a new method.\\

\subsection{Digital Image Watermarking Based on DCT}
Discrete cosine transform (DCT) based watermarking methods is the currently most extensive used method in the area of steganography. DCT transformation like fourier transform, intends to reduce existed redundancy in the frequency domain. DCT is a summation of a finite numbers of cosine functions in different frequencies, which represents the identification of processed signal. To embed the watermark message it to the DCT coefficients, we first divide cover image to patches and calculate DCT transform and preparing a stream from DCT coefficients. 
An image can be decomposed into low-frequency, mid-frequency and high frequency parts refer to their DCT coefficients. We prefer to embed the the watermark message in DCT coefficients by adding the modulated message multiplied by $\alpha$ to the mid-frequency coefficients of cover image DCT coefficeitns. $\alpha$ controls the perceptual quality of host image in order to get a powerful watermark in encountering with types of attacks. Finally we can reconstruct the watermarked image by IDCT. To extract the watermark, same procedure is applied as the insertion steps. Besides, the original host image is needed to help us to detect the embedded watermark via comparing coefficients obtained with coefficients of original image without watermark. \\


\subsection{Proposed Watermarking Scheme Based on MCA}
Because human eye is insensitive to noise in high frequency texture parts
Inspired my \textit{Morphological Component Analysis}, a novel blind digital watermark scheme based on MCA, which contains both watermark embedding and extraction process is proposed. The highlight of this methods is that MCA is better at separating the image layers by distinctive morphologies. As mentioned in the introduction part. A image can be decomposed into texture and contour parts, where the former mainly contains high-frequency information (e.g. ripple or cloth), the latter contains low-frequency information. Textures can typically fragile to attacks such as JPEG compression, cropping, rotating or additive noise. Hence we prefer to modulate the watermark information in mid/high-frequency parts of the cover image.\\

Next we explains each step of this method. Decompose the image into layers of texture and contour using curvelet transform and local discrete cosine transform as dictionaries.

\subsection{Overall Process and Performance}
In the last section we described two digital watermarking techniques. By applying each one of them to a $512 \times 512$ gray-scale `lena' image, we are able to evaluate their extraction quality respectively, also the robustness under different types of attacking.  

